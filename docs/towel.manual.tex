\documentclass{book}

\usepackage[left=1in, right=1in, top=1in, bottom=1in]{geometry}
\usepackage{hyperref}

\title{Towel Reference Manual}

\begin{document}
\maketitle
\tableofcontents

\chapter{Lexical Elements}\label{chap:grammar}
\section{Keywords}

Keywords in the Towel programming language is as follows:
\begin{verbatim}
if>=0 if>0 if<=0 if<0 if~0 if=0 ift iff ife ifne
match fun bind type also then
\end{verbatim}

The corresponding tokens are:
\begin{verbatim}
IFGEZ IFGZ IFLEZ IFLZ IFNEZ IFEZ IFT IFF IFE IFNE
MATCH FUNCTION BIND TYPE ALSO THEN
\end{verbatim}

\section{Punctuations}

Punctuations used in the Towel programming language are as follows:

\begin{itemize}
\item Whitespace characters are simply ignored.
\item These characters have special meanings in the Towel programming language: \texttt{` ' `` , ; . ( ) [ ] { } \ @ eof}. This means that you cannot use these characters in names and atoms. \footnote{In other words, you can use any other punctuation characters in names and atoms.}
\item Any unprintable character is reserved and won't be used.
\end{itemize}

Specially,
\begin{verbatim}
  TERMINATOR ::= ['.' '\n' '\r' '\r\n' eof]
\end{verbatim}

\section{Names}

Names are used for naming (or referencing to) values. Valid names should not start with reserved punctuations, lowercased letters, and numbers.

More formally,
\begin{verbatim}
  reserved_punct ::= ['`' ''' '"' ',' ';' '.' '\' '@' 
                      '(' ')' '[' ']' '{' '}' '\n' '\r' ' ' '\t']
  valid_punct ::= ['!' '~' '#' '$' '%' '^' '&' '*' '-' '_' '+' '=' '|'
                   ':' '<' '>' '?' '/']
  BQUOTE ::= '`'
  SQUOTE ::= '''
  DQUOTE ::= '"'
  COMMA ::= ','
  SEMICOLON ::= ';'
  PERIOD ::= '.'
  SLASH ::= '\'
  AT ::= '@'
  LPAREN ::= '('
  RPAREN ::= ')'
  LBRACKET ::= '['
  RBRACKET ::= ']'
  LBRACE ::= '{'
  RBRACE ::= '}'

  digit ::= ['0'-'9']
  lc_chars ::= ['a'-'z']
  NAME ::= [^ '-' reserved_punct digit lc_chars] [^ reserved_punct]*
\end{verbatim}

\section{Comments}

Comments are defined as follows:
\begin{verbatim}
  __COMMENTS ::= '"' [^ '"']* '"'
\end{verbatim}

When the scanner encounters any other character not mentioned above, it will raise a \texttt{LexicalError} exception.

\chapter{Data Types}
\label{chap:data-types}

Types are important and inferred in Towel. If the type check fails, the compiler will refuse to compile the program.

\section{Primitive Types}
Towel provides to the user the following primitive built-in types:
\begin{itemize}
\item Atom
\item FixedInt
\item Int (won't be implemented until later versions)
\item String
\item Float
\item List
\item Tuple
\end{itemize}

\subsection{Atoms}
Atoms are special names uniquely bound to constants. They are of type \texttt{atom}.

\subsection{\texttt{FixedInt}s, \texttt{Int}s, \texttt{Float}s}
Fixed integers and floats are simply OCaml integers and floats. \texttt{Int}s are integers of arbitrary precision (like those \texttt{int}s in Python). These types are subclass of the class \texttt{Number}. A lot of arithmetic functions take \texttt{Number}s as their arguments. However, bitwise functions will take \texttt{FixedInt}s as arguments.

\subsection{Strings}
Strings are implemented as OCaml ones. String items are surrended by single quote, rather than double quote.\footnote{Because you don't have to hit the \textit{shift} key when inputing single quotes. Same goes for brackets.}

\subsection{Lists and Tuples}
Lists and tuples are implemented as OCaml lists. Types of tuples, for instance, \texttt{PT\_Tuple\-1} for 1-element tuples, are decided at compile time, and cannot be changed later, whereas the lengths of lists are not fixed.

\section{\texttt{PT\_Any}}
In current version, without generic typing and algebraic types implemented, \texttt{PT\_Any} is the superclass of the type of every value in Towel.

\section{Algebraic Data Types}
Users can use the \texttt{type} special form to define custom algebraic data types, see also \autoref{ssec:atdf} and \autoref{chap:examples}.

\section{Functions}
The type of functions is a list of types, which consists of the types of input arguments along with the return type. When applying arguments to functions, if the arguments are not enough (i.e. the caller's stack exhausts before all the arguments are satisfied), a partial applied function is returned.\footnote{Maybe we could call this as partial Currying?}

\chapter{Elements of Programs}
\label{chap:forms}

\section{Basic Concepts}

A Towel program consists of one or multiple so-called \textbf{words} and one terminator. A terminator can be a period (dot) or the end-of-file character.

When encountered multiple words, they are always evaluated one by one in the order they appear.

A word, can be one of the following:
\begin{itemize}
\item literal
\item name
\item sequence
\item backquote
\item \texttt{match} and \texttt{if} forms
\item function form
\item bind form
\item type declaring form
\item at special infix form
\end{itemize}

\section{Rules for evaluation}

When evaluating a Towel program, each word is evaluated in the original order:
\begin{itemize}
\item For integer, float, string, and atom literals, return references to them directly;
\item For each list literal, return a reference to the list after you have evaluated each item of the list;
\item For a backqoute, return whatever is quoted (without evaluating);
\item For a sequence, create a new function out of the sequence and evaluate that function (i.e. evaluate the sequence on a new stack) and return the TOS of the new stack;
\item For a function, apply\footnote{A stack is always created along with a function invocation, this is to avoid the potential corruption of shared stacks.} required arguments to it (if not, a backquoted partial applied function will be returned), then evaluate the return value of the function with the caller's stack and return the evaluated value;
\item For a name, look up the value it references to, evaluated that value and return the evaluated value.
\item For \texttt{match} and \texttt{if} forms, match or test against the TOS and evaluate the word in the matched branch;
\item For bind-then form, evaluate the value that gets bound and bind the value evaluated to the name, then evaluate the then section with current scope.
\end{itemize}

After evaluating the words, you push the result onto the stack.

\section{Literal}

A literal is a literal value whose type is of the data types we have talked about in \autoref{chap:data-types}.

\subsection{Atoms}

You can create atoms by writing any lowercased letter followed by arbitrary length of characters that are not reserved punctuations and keywords.

Atoms are unique across the entire program. Because atoms are assigned with a unique integer, so you can have no more than $2^{31}$ of them in your program.


Atoms are good for pattern matching.\footnote{This idea is from Erlang.} Note that bool types are implemented as atoms (\texttt{true\textbackslash Bool} and \texttt{false\textbackslash Bool}).

\subsection{Numbers and Strings}

You can create number and string literals by writing like this:
\begin{verbatim}
  1 -1 2 -3 5 -8 13 -21
  1.1e1 -0.1 'don\'t panic'
\end{verbatim}

\subsection{Lists}

When creating a list literal, you must write a list of words separated with spaces in a pair of brackets, like the following code:
\begin{verbatim}
  [arthur-dent ford-prefect betelgeuse]
  [Spam Spam Spam]
  [Spam ifne (More Spam)`, (Less Spam)`]
\end{verbatim}

Note that in the last list literal, there exists an \texttt{ifne} form. This
is totally valid, because \texttt{if} forms are also words, so as long as the \textbf{type} of the value the \texttt{ifne} form evaluates to match, you are good. And you should be aware that \texttt{if} forms always test against the top element of current data stack, so the name \texttt{Spam} before \texttt{ifne} has nothing to do with the value the \texttt{ifne} form evaluates to.

\subsection{Tuples}

Tuples are fixed length lists, whereas you can append new items to lists to create new lists. Create tuples like this:
\begin{verbatim}
  [\ arthur-dent ford-prefect betelgeuse] "a 3-tuple of type PT_Tuple-3"
  [\] "a 0-tuple of type PT_Tuple-0"
\end{verbatim}

You can use tuple in match forms like this:
\begin{verbatim}
  match [\ a b A], A
\end{verbatim}
This match form matches a 3-tuple and if the first and second element are atom \texttt{a} and \texttt{b} respectively, it binds the third element to the name \texttt{A}. More on match, see also TODO:ref:match.

\subsection{Literal of Algebraic Type}

You can create literals of algebraic type by listing the required parameters in a pair of backet with an at symbol followed by the left bracket. And creating constructors by concatenating the constructor atom and the type name with a slash. For instance\footnote{I know this seems ridiculous. But they are (and was) great comedians.}:
\begin{verbatim}
  [@ chapman
    [@ cleese
      [@ gilliam
        [@ idle
          [@ jones
            [@ palin nil\List]
             cons\List]
           cons\List]
         cons\List]
       cons\List]
     cons\List]
   cons\List
\end{verbatim}

\section{Sequence}

Sequences are short-hand forms for creating anonymous functions with no arguments. You can create a sequence by writing the function body between a pair of parenthesis.

Towel also provides another kind of sequences, the shared sequences. These kind of sequences share the same stack with the caller. When creating such sequences, you add an at symbol right after the left parenthesis. For example,
\begin{verbatim}
  ((A B - if>0 1, 0) (A B + if<0 2, 3) Bitand) "non-shared regular sequences"
  (A B - (@ if>0, 1, 0) Println) "prints 1 or 0"
\end{verbatim}

\section{Backquote}

Because Towel evaluates and pushes everything it encounters, you can use backquotes the values to prevent Towel from evaluating them so that Towel pushes them directly onto the data stack. Backquotes are created by appending a backquote to the values you want to backquote.

You can backquote only limit types of values:
\begin{itemize}
\item literals
\item names
\item sequences
\item backquotes
\end{itemize}

You can create backquoted shared sequence by replacing the parentheses with braces and dropping the at symbol. See also \autoref{ssec:backquote} and \autoref{ssec:macro}.

\section{\texttt{match} and \texttt{if} forms}

1

\section{Function form}

1

\section{Bind form}

1

\section{Algebraic Data Type Declaration}
\label{ssec:atdf}

Algebraic data types will be basically the same with that of OCaml except the syntax is different. The syntax here in Towel is designed in a postfix fashion. First, use \texttt{type NAME} to bind the type you are declaring to the name \texttt{NAME}. Then declare constructors using atoms. Before the constructor, put the types of parameters in a list with an at symbol in the beginning. If the types of parameters are parameterized, put the parameters in a pair of braces before the type, like this:
\begin{verbatim}
  type Option [@ a] some, none
  type List [@ a {a}List] cons, nil
  type BinTree [@ {a}BinTree {a}BinTree] tree, [@ a] leaf
\end{verbatim}

You can only put atoms, parameterized names (potential namespaced) into the parameter list. See also \autoref{ssec:adt-example}.

\chapter{Examples}
\label{chap:examples}

The \texttt{traverse} tool produces correct syntax tree with these examples.

\section{Concrete examples}
\subsection{Quicksort}
\begin{verbatim}
  bind Quicksort fun L,
    match
      [], [];
      Head Tail ::,
        (Tail (Head <) Filter Quicksort
         [Head]
         Tail (Head >=) Filter Quicksort
         ++3)
  then ([5 4 3 2 1] Quicksort).
\end{verbatim}

\footnote{In order to eliminate the shift/reduce conflicts, I have to import a lot of parentheses (sequences).}Note that \texttt{++3} is a trinary version of function \texttt{++} (list concatenation).

Also The second pattern action is written as a sequence, which creates an anonymous function whose body is the forms in the sequence, then the anonymous function is evaluted. The scope is shared between these functions, so \texttt{Head} and \texttt{Tail} are still visible to the anonymous function.

\subsection{Greatest common divisor}
\begin{verbatim}
  bind Greatest-common-divisor fun X(Int) Y,
    (- if=0 X,
       if>0 (Y X - Y Greatest-common-divisor),
            (X X Y - Greatest-common-divisor))
  then (42 24 Greatest-common-divisor).
\end{verbatim}

Note that \texttt{X} and \texttt{Y} are already in the stack by default (because Towel pushes arguments onto stack), so we can immediately evaluate function \texttt{-}.

\subsection{Fibonacci numbers}
\begin{verbatim}
  bind Fib fun A B N,
    if=0 A, (A B + A 1 N - Fib)
  then (1 1 10 Fib).
\end{verbatim}

\subsection{Something about backquotes}
\label{ssec:backquote}
\begin{verbatim}
  bind SomeFun fun A,
    if~0 +`, -`
  then bind AnotherFun fun A B,
    (A B A SomeFun)
  then (1 5 AnotherFun).
\end{verbatim}

A quick explanation: when \texttt{SomeFun} is called with \texttt{A}, it returns either evaluated backquoted name \texttt{+} or \texttt{-}, in other words, it returns either name \texttt{+} or \texttt{-}.

What the \textbf{returning} actually does is that it cleans up the current function, and pushes whatever is on top of the stack (a name \texttt{+} or \texttt{-} in this case) of the current function (in this case, it is the stack of \texttt{SomeFun}) onto the caller's stack (in this case, it is the stack of \texttt{AnotherFun}). And finally jump to the instruction next to the last one. So there is a name \texttt{+} or \texttt{-} on top of the function \texttt{AnotherFun}.

And because we are evaluating return values of functions, \texttt{+} and \texttt{-} are evaluated (derefenced to) some function values, for example \texttt{fv1:0x0001} and \texttt{fv2:0x0002}. And again because we are evaluating values that the names are pointing to, one of \texttt{fv1:0x0001} and \texttt{fv2:0x0002} is called\footnote{It is worth mentioning that because we have exited \texttt{SomeFun}, we are now evaluating the function value with the stack of \texttt{AnotherFun}} with \texttt{A} and \texttt{B}, thus resulting in either adding or substracting \texttt{A} and \texttt{B}.

\section{Advanced examples}
\subsection{Macro}
\label{ssec:macro}
\begin{verbatim}
  bind A-Macro fun,
    (@ if~0 +, -)`
  also B-Macro fun,
    {if~0 +, -}
  then bind AnotherFun fun A B,
    (A B A-Macro)
  then (1 2 AnotherFun).
\end{verbatim}

A quick explanation: \texttt{(@ if~0 +, -)} is returned as we want, so a sequence (which is essentially an anonymous function that test whether the value on top of the stack is zero), and by adding a \texttt{@} we define the sequence to be a shared sequence which shares the same stack with the caller, so the overall effect is like we have done a code replacement.

\texttt{B-Macro} is a short-hand version of \texttt{A-Macro}

\subsection{Algebraic Data Type}
\label{ssec:adt-example}
\begin{verbatim}
  type BinTree [@ {a}BinTree {a}BinTree] tree,
               [@ a] leaf
  also MyList [@ a {a}MyList] cons,
              nil
  then bind A [@ [@ [@ 42 nil\MyList] cons\MyList] leaf\BinTree
                 [@ [@ 42 nil\MyList] cons\MyList] leaf\BinTree]
              tree\BinTree
  then (A export).
\end{verbatim}

This is equivalent to the following OCaml code:
\begin{verbatim}
  type 'a bin_tree = Tree of a bin_tree * a bin_tree
                   | Leaf of a;;
  type 'a my_list = Cons of a * a my_list
                  | Nil;;
  let a = Tree(Leaf(Cons(42, Nil)), Leaf(Cons(42, Nil)))
\end{verbatim}

\end{document}