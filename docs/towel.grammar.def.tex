\documentclass{article}

\usepackage[left=1in, right=1in, top=1in, bottom=1in]{geometry}

\title{Towel Grammar Definition}

\begin{document}
\maketitle
\tableofcontents

\section{Syntactic definition}

\texttt{menhir} compiles the grammar below without complaining about shift/reduce conflicts.

A Towel program is basically a \texttt{sentence} consisting a series of \texttt{word}s and a \texttt{TERMINATOR} token.
\begin{verbatim}
  sentence: word* TERMINATOR
\end{verbatim}

\subsection{Words}
\label{ssec:words}

A \texttt{word} (or form) is a basic element of Towel programs:
\begin{verbatim}
  word: literal
      | name
      | sequence
      | backquote
      | control_sequence
      | function
      | bind_sform
      | type_decl_sform
\end{verbatim}

\subsection{Literals}

Although list literals are, like strings, delimited by delimiters (single quotes for strings, brackets for lists). However, string items are simply characters that can be handled by scanner, while list items are much more complicated than characters. Thus list literals must be parsed by parser:
\begin{verbatim}
  list_literal: LBRACKET word* RBRACKET
  tuple_literal: LBRACKET SLASH word* RBRACKET

  altype_literal_constructor: ATOM SLASH name
  altype_literal_item: word
                     | altype_literal_constructor
  altype_literal: LBRACKET AT altype_literal_item BRACKET

  literal: LITERAL (* LITERAL tokens from scanner *)
         | ATOM (* ATOM tokens from scanner *)
         | list_literal
         | tuple_literal
         | altype_literal
\end{verbatim}

\subsection{Names}

Referencing to names in other namespaces (or modules) is allowed with the following rule:
\begin{verbatim}
  name: NAME SLASH name
      | NAME
      | name AT
\end{verbatim}

The last rule is for explicitly tag a function call as tail-recursive.

\subsection{Sequences}

A sequence is a series of words delimited by a pair of parentheses. They are basically anonymous functions that use the same stack along with the current function. See example section for detailed use cases. The rule for parsing sequences is as follows:
\begin{verbatim}
  sequence: LBRACE word* RBRACE
          | LBRACE AT word* RBRACE
\end{verbatim}

The second rule parses sequences as shared sequences (they share the same stack with the caller, useful when implementing macros).

\subsection{Backquotes}

A \texttt{backquote} is a special form that defers the evaluation of names and values. You can backquote any value and name. You can quote sequences as well.
\begin{verbatim}
  backquote: literal BQUOTE
           | name BQUOTE
           | sequence BQUOTE
           | backquote BQUOTE
           | LBRACE word* RBRACE
\end{verbatim}

Note that \texttt{{ word* }} will create a backquoted shared sequenced, which is extremely useful when creating (runtime) macros.

\subsection{Control sequences}

Control sequences consists of \texttt{if} special forms and \texttt{match} speical form:
\begin{verbatim}
  control_sequence: if_sform
                  | match_sform

  if_sform: IFGEZ word COMMA word
          | IFGZ word COMMA word
          | IFLEZ word COMMA word
          | IFLZ word COMMA word
          | IFE word COMMA word
          | IFNE word COMMA word
          | IFEZ word COMMA word
          | IFNEZ word COMMA word
          | IFT word COMMA word
          | IFF word COMMA word

  match_sform: MATCH (pattern SEMICOLON)* pattern
  (* e.g. "match pattern1, form1; pattern2, form2" *)

  pattern: word* COMMA restricted_word (* e.g. "pattern, form" *)

  restricted_word: name
                 | backquote
                 | literal
                 | sequence
  (* this rule is to avoid some shift/reduce conflicts *)
\end{verbatim}

The rationale of providing so many \texttt{if} forms is to support chained \texttt{if} forms naturally, like this:
\begin{verbatim}
  A if>0 Something,
    if<0 (Some other things),
    if=0 (Some more other things),
      Impossible.
\end{verbatim}

rather than:
\begin{verbatim}
  0 A > ift Something,
 (0 A < ift (Some other things),
 (0 A = ift (Some more other things),
          Impossible)).
\end{verbatim}

\subsection{Functions}

Function forms are used to define functions:
\begin{verbatim}
  function: FUNCTION arg_def* COMMA word

  arg_def: NAME
         | NAME type_def

  type_def: LPAREN name+ RPAREN
\end{verbatim}

\subsection{Bind-Then forms}

Bind-Then forms are special forms about name scoping. They are defined as follows:
\begin{verbatim}
  bind_sform: BIND NAME word (ALSO NAME word)* THEN word
\end{verbatim}

With token \texttt{ALSO}, you can bind multiple values to multiple names simultaneous, enabling you to cross-reference these names.

\subsection{Type Declaration Forms}\label{ssec:tdsf}

In order to support\footnote{Implementation for algebraic data types won't be around until later versions.} custom type declaration, a kind of special form is provided as follows:
\begin{verbatim}
  altype_param: LBRACE ATOM+ RBRACE
  altype_case_def_item: ATOM
                      | NAME
                      | NAME altype_param
  altype_case_def: (LBRACKET AT altype_case_def_item* RBRACKET)? ATOM
  altype_def: NAME altype_case_def (COMMA altype_case_def)*
  altype_sform: TYPE altype_def (ALSO altype_def)* THEN word
\end{verbatim}

\end{document}